\section{Protocolos de Seguridad (ISO 27001 / 9001)}

\subsection{Control de Acceso (RBAC)}
El sistema implementa un control de acceso basado en roles estricto:

\begin{enumerate}
    \item \textbf{Anonimo:} Lectura básica. Rate Limit: 100/min.
    \item \textbf{Registrado:} Lee capítulos. Rate Limit: 1000/min.
    \item \textbf{Verificado (+18):} Acceso NSFW via \texttt{is\_nsfw\_allowed}.
    \item \textbf{Staff:} Permisos de escritura y moderación.
\end{enumerate}

\subsection{Autenticación}
Se utiliza el estándar **JWT (JSON Web Token)**.
\begin{itemize}
    \item \textbf{Access Token:} Vida útil de 60 minutos.
    \item \textbf{Refresh Token:} Vida útil de 7 días. Rotación automática.
\end{itemize}

\subsection{Protección de Infraestructura}
\subsubsection{Rate Limiting}
Para mitigar ataques DDoS y scraping abusivo:
\begin{lstlisting}[language=python, caption=Configuración de Throttling]
'DEFAULT_THROTTLE_RATES': {
    'anon': '100/min',
    'user': '1000/min'
}
\end{lstlisting}

\subsubsection{Cabeceras de Seguridad}
Se fuerzan cabeceras HTTP para protección del cliente:
\begin{itemize}
    \item \texttt{SECURE\_BROWSER\_XSS\_FILTER}: Previene ataques XSS reflejados.
    \item \texttt{SECURE\_CONTENT\_TYPE\_NOSNIFF}: Bloquea sniffing de MIME types.
\end{itemize}

\subsection{Auditoría}
Todas las acciones administrativas y de acceso sensible se registran a través del middleware \texttt{DACAuditMiddleware}, almacenando IP, Usuario y Recurso accedido.
