\section{Manual de Operaciones y Despliegue}

\subsection{Ciclo de Vida del Software (SDLC)}
El desarrollo sigue un flujo de CI/CD (Integración y Despliegue Continuo):
1. \textbf{Desarrollo Local:} Feature branches. Tests locales con `manage.py check`.
2. \textbf{Push a Main:} GitHub Actions o Webhooks disparan el build en Railway.
3. \textbf{Build:} Se instalan dependencias (`pip install -r requirements.txt`).
4. \textbf{Pre-Deploy:} Se ejecutan migraciones de base de datos (`manage.py migrate`).
5. \textbf{Live:} El nuevo contenedor reemplaza al anterior sin tiempo de inactividad (Zero Downtime).

\subsection{Monitorización y Alertas}
\begin{itemize}
    \item \textbf{Disponibilidad:} Endpoint `/api/health/` consultado por Cloudflare cada 60s.
    \item \textbf{Errores:} Integración con Sentry. Notificación inmediata de excepciones 500/compatibilidad.
    \item \textbf{Logs:} Accesibles vía CLI de Railway o Dashboard Web.
\end{itemize}

\subsection{Procedimientos de Recuperación (Disaster Recovery)}
\subsubsection{Restauración de Servicio}
En caso de fallo crítico en el despliegue:
1. Acceder al Dashboard de Railway.
2. Seleccionar el despliegue anterior ("Rollback").
3. El sistema volverá a la versión estable en <30 segundos.

\subsubsection{Backup de Datos}
La base de datos MySQL en Railway tiene backups automáticos diarios con retención de 7 días.
