\documentclass[12pt,a4paper]{article}

\usepackage[utf8]{inputenc}
\usepackage[T1]{fontenc}
\usepackage{lmodern}
\usepackage[spanish]{babel}
\usepackage{geometry}
\usepackage{hyperref}
\usepackage{listings}
\usepackage{xcolor}
\usepackage{titlesec}

\geometry{
    top=2.5cm,
    bottom=2.5cm,
    left=2.5cm,
    right=2.5cm
}

% Code Style
\definecolor{codegreen}{rgb}{0,0.6,0}
\definecolor{codegray}{rgb}{0.5,0.5,0.5}
\definecolor{codepurple}{rgb}{0.58,0,0.82}
\definecolor{backcolour}{rgb}{0.95,0.95,0.92}

\lstdefinestyle{mystyle}{
    backgroundcolor=\color{backcolour},   
    commentstyle=\color{codegreen},
    keywordstyle=\color{magenta},
    numberstyle=\tiny\color{codegray},
    stringstyle=\color{codepurple},
    basicstyle=\ttfamily\footnotesize,
    breakatwhitespace=false,         
    breaklines=true,                 
    captionpos=b,                    
    keepspaces=true,                 
    numbers=left,                    
    numbersep=5pt,                  
    showspaces=false,                
    showstringspaces=false,
    showtabs=false,                  
    tabsize=2
}

\lstdefinelanguage{json}{
    basicstyle=\normalfont\ttfamily,
    numbers=left,
    numberstyle=\scriptsize,
    stepnumber=1,
    numbersep=8pt,
    showstringspaces=false,
    breaklines=true,
    frame=lines,
    backgroundcolor=\color{backcolour},
    literate=
     *{0}{{{\color{codepurple}0}}}{1}
      {1}{{{\color{codepurple}1}}}{1}
      {2}{{{\color{codepurple}2}}}{1}
      {3}{{{\color{codepurple}3}}}{1}
      {4}{{{\color{codepurple}4}}}{1}
      {5}{{{\color{codepurple}5}}}{1}
      {6}{{{\color{codepurple}6}}}{1}
      {7}{{{\color{codepurple}7}}}{1}
      {8}{{{\color{codepurple}8}}}{1}
      {9}{{{\color{codepurple}9}}}{1}
      {:}{{{\color{codegray}{:}}}}{1}
      {,}{{{\color{codegray}{,}}}}{1}
      {\{}{{{\color{codegreen}{\{}}}}{1}
      {\}}{{{\color{codegreen}{\}}}}}{1}
      {[}{{{\color{codegreen}{[}}}}{1}
      {]}{{{\color{codegreen}{]}}}}{1},
}

\lstset{style=mystyle}

\title{Documentación API MangaReader}
\author{Equipo de Desarrollo}
\date{\today}

\begin{document}

\maketitle

\section*{Control de Documentos (ISO 9001)}
\begin{center}
\begin{tabular}{|c|c|c|c|}
    \hline
    \textbf{Versión} & \textbf{Fecha} & \textbf{Autor} & \textbf{Descripción} \\
    \hline
    1.0 & \today & Equipo Dev & Versión Inicial del Sistema \\
    \hline
\end{tabular}
\end{center}

\tableofcontents
\newpage

\section{Arquitectura del Sistema}

\subsection{Visión General}
MangaReader es una plataforma de distribución de contenido digital (manga/manhwa) construida sobre una arquitectura monolítica modular con Django REST Framework. El sistema está diseñado para alta disponibilidad, escalabilidad horizontal y seguridad robusta.

\subsection{Tecnologías Principales}
\begin{itemize}
    \item \textbf{Backend:} Python 3.13 + Django 5.2.
    \item \textbf{API:} Django REST Framework (DRF) con JWT.
    \item \textbf{Base de Datos:} MySQL (Producción), SQLite (Desarrollo).
    \item \textbf{Infraestructura:} Railway (PaaS) con despliegue continuo desde GitHub.
    \item \textbf{Monitorización:} Sentry (Errores), Cloudflare (Salud/DNS).
\end{itemize}

\subsection{Diagrama de Componentes}
El sistema se divide en los siguientes módulos lógicos dentro de \texttt{ApiCore}:
\begin{enumerate}
    \item \textbf{Manga Core:} Gestión de títulos, sinopsis, estados (Modelos: \texttt{manga}, \texttt{manga\_cover}).
    \item \textbf{Capítulos:} Gestión de contenido multimedia y volúmenes (Modelo: \texttt{chapter}).
    \item \textbf{Usuarios y Seguridad:} Sistema RBAC (Role-Based Access Control) extendiendo \texttt{auth.User} con \texttt{UserProfile}.
    \item \textbf{Mantenedores:} Tablas maestras para Tags, Autores, Demografías.
    \item \textbf{DAC (Digital Access Control):} Auditoría y control de acceso granular.
\end{enumerate}

\subsection{Flujo de Datos}
1. El cliente (Frontend/App) solicita un token JWT vía \texttt{/api/token/}.
2. Las peticiones subsiguientes incluyen el header \texttt{Authorization: Bearer <token>}.
3. El Middleware de Seguridad (\texttt{DACAuditMiddleware}) intercepta la petición para registro.
4. El \texttt{MangaViewSet} evalúa los permisos (\texttt{CanViewNSFW}, \texttt{IsAuthenticated}).
5. El Serializador (\texttt{MangaCardSerializer}) transforma los datos y optimiza la respuesta.

\section{Base de Datos y Modelos}

\subsection{Esquema Relacional}
El sistema utiliza un esquema relacional normalizado para garantizar la integridad de los datos.

\subsubsection{Modelo de Usuarios (\texttt{user\_models})}
\begin{itemize}
    \item \textbf{UserProfile}: Extensión 1-a-1 de \texttt{auth.User}. Almacena:
    \begin{itemize}
        \item \texttt{is\_nsfw\_allowed}: Booleano para control parental.
        \item \texttt{reputation}: Sistema de karma/confianza.
        \item \texttt{role}: Rol funcional (Lector, Moderador, Admin).
    \end{itemize}
\end{itemize}

\subsubsection{Modelo de Mangas (\texttt{manga\_models})}
La entidad central \texttt{manga} se relaciona con:
\begin{itemize}
    \item \textbf{Estado:} FK a \texttt{estados} (En emisión, Finalizado).
    \item \textbf{Demografía:} FK a \texttt{demografia} (Seinen, Shonen).
    \item \textbf{Tags:} M2M a \texttt{tags} para categorización.
    \item \textbf{Autores:} M2M a \texttt{autores}.
\end{itemize}

\subsubsection{Modelo de Estructura (\texttt{mantenedor\_models})}
Tablas auxiliares para estandarización:
\begin{itemize}
    \item \texttt{demografia}: Clasificación de audiencia.
    \item \texttt{estados}: Ciclo de vida del contenido.
    \item \texttt{idioma}: Soporte multi-lenguaje.
\end{itemize}

\subsection{Políticas de Integridad}
\begin{itemize}
    \item \textbf{Foreign Keys:} Se utiliza \texttt{on\_delete=models.PROTECT} en catálogos maestros para evitar borrados accidentales de categorías en uso.
    \item \textbf{Transacciones:} Escritura crítica usa \texttt{transaction.atomic}.
\end{itemize}

\section{Protocolos de Seguridad (ISO 27001 / 9001)}

\subsection{Control de Acceso (RBAC)}
El sistema implementa un control de acceso basado en roles estricto:

\begin{enumerate}
    \item \textbf{Anonimo:} Lectura básica. Rate Limit: 100/min.
    \item \textbf{Registrado:} Lee capítulos. Rate Limit: 1000/min.
    \item \textbf{Verificado (+18):} Acceso NSFW via \texttt{is\_nsfw\_allowed}.
    \item \textbf{Staff:} Permisos de escritura y moderación.
\end{enumerate}

\subsection{Autenticación}
Se utiliza el estándar **JWT (JSON Web Token)**.
\begin{itemize}
    \item \textbf{Access Token:} Vida útil de 60 minutos.
    \item \textbf{Refresh Token:} Vida útil de 7 días. Rotación automática.
\end{itemize}

\subsection{Protección de Infraestructura}
\subsubsection{Rate Limiting}
Para mitigar ataques DDoS y scraping abusivo:
\begin{lstlisting}[language=python, caption=Configuración de Throttling]
'DEFAULT_THROTTLE_RATES': {
    'anon': '100/min',
    'user': '1000/min'
}
\end{lstlisting}

\subsubsection{Cabeceras de Seguridad}
Se fuerzan cabeceras HTTP para protección del cliente:
\begin{itemize}
    \item \texttt{SECURE\_BROWSER\_XSS\_FILTER}: Previene ataques XSS reflejados.
    \item \texttt{SECURE\_CONTENT\_TYPE\_NOSNIFF}: Bloquea sniffing de MIME types.
\end{itemize}

\subsection{Auditoría}
Todas las acciones administrativas y de acceso sensible se registran a través del middleware \texttt{DACAuditMiddleware}, almacenando IP, Usuario y Recurso accedido.

\section{Manual de Operaciones y Despliegue}

\subsection{Ciclo de Vida del Software (SDLC)}
El desarrollo sigue un flujo de CI/CD (Integración y Despliegue Continuo):
1. \textbf{Desarrollo Local:} Feature branches. Tests locales con `manage.py check`.
2. \textbf{Push a Main:} GitHub Actions o Webhooks disparan el build en Railway.
3. \textbf{Build:} Se instalan dependencias (`pip install -r requirements.txt`).
4. \textbf{Pre-Deploy:} Se ejecutan migraciones de base de datos (`manage.py migrate`).
5. \textbf{Live:} El nuevo contenedor reemplaza al anterior sin tiempo de inactividad (Zero Downtime).

\subsection{Monitorización y Alertas}
\begin{itemize}
    \item \textbf{Disponibilidad:} Endpoint `/api/health/` consultado por Cloudflare cada 60s.
    \item \textbf{Errores:} Integración con Sentry. Notificación inmediata de excepciones 500/compatibilidad.
    \item \textbf{Logs:} Accesibles vía CLI de Railway o Dashboard Web.
\end{itemize}

\subsection{Procedimientos de Recuperación (Disaster Recovery)}
\subsubsection{Restauración de Servicio}
En caso de fallo crítico en el despliegue:
1. Acceder al Dashboard de Railway.
2. Seleccionar el despliegue anterior ("Rollback").
3. El sistema volverá a la versión estable en <30 segundos.

\subsubsection{Backup de Datos}
La base de datos MySQL en Railway tiene backups automáticos diarios con retención de 7 días.


\section{Referencia API}
Esta sección detalla los endpoints operativos para integración.

\subsection{Autenticación}
\begin{lstlisting}[language=bash]
POST /api/token/
\end{lstlisting}
Body:
\begin{lstlisting}[language=json]
{
    "username": "usuario",
    "password": "password"
}
\end{lstlisting}

\subsection{Mangas}
\begin{lstlisting}[language=bash]
GET /api/manga/mangas/?limit=20
GET /api/manga/mangas/{id}/
\end{lstlisting}

\subsection{Monitorización}
\begin{lstlisting}[language=bash]
GET /api/health/
\end{lstlisting}

\end{document}
